\documentclass[]{article}
\usepackage{lmodern}
\usepackage{amssymb,amsmath}
\usepackage{ifxetex,ifluatex}
\usepackage{fixltx2e} % provides \textsubscript
\ifnum 0\ifxetex 1\fi\ifluatex 1\fi=0 % if pdftex
  \usepackage[T1]{fontenc}
  \usepackage[utf8]{inputenc}
\else % if luatex or xelatex
  \ifxetex
    \usepackage{mathspec}
  \else
    \usepackage{fontspec}
  \fi
  \defaultfontfeatures{Ligatures=TeX,Scale=MatchLowercase}
\fi
% use upquote if available, for straight quotes in verbatim environments
\IfFileExists{upquote.sty}{\usepackage{upquote}}{}
% use microtype if available
\IfFileExists{microtype.sty}{%
\usepackage{microtype}
\UseMicrotypeSet[protrusion]{basicmath} % disable protrusion for tt fonts
}{}
\usepackage[margin=1in]{geometry}
\usepackage{hyperref}
\hypersetup{unicode=true,
            pdftitle={fav things demo},
            pdfauthor={Jen Richmond},
            pdfborder={0 0 0},
            breaklinks=true}
\urlstyle{same}  % don't use monospace font for urls
\usepackage{color}
\usepackage{fancyvrb}
\newcommand{\VerbBar}{|}
\newcommand{\VERB}{\Verb[commandchars=\\\{\}]}
\DefineVerbatimEnvironment{Highlighting}{Verbatim}{commandchars=\\\{\}}
% Add ',fontsize=\small' for more characters per line
\usepackage{framed}
\definecolor{shadecolor}{RGB}{248,248,248}
\newenvironment{Shaded}{\begin{snugshade}}{\end{snugshade}}
\newcommand{\AlertTok}[1]{\textcolor[rgb]{0.94,0.16,0.16}{#1}}
\newcommand{\AnnotationTok}[1]{\textcolor[rgb]{0.56,0.35,0.01}{\textbf{\textit{#1}}}}
\newcommand{\AttributeTok}[1]{\textcolor[rgb]{0.77,0.63,0.00}{#1}}
\newcommand{\BaseNTok}[1]{\textcolor[rgb]{0.00,0.00,0.81}{#1}}
\newcommand{\BuiltInTok}[1]{#1}
\newcommand{\CharTok}[1]{\textcolor[rgb]{0.31,0.60,0.02}{#1}}
\newcommand{\CommentTok}[1]{\textcolor[rgb]{0.56,0.35,0.01}{\textit{#1}}}
\newcommand{\CommentVarTok}[1]{\textcolor[rgb]{0.56,0.35,0.01}{\textbf{\textit{#1}}}}
\newcommand{\ConstantTok}[1]{\textcolor[rgb]{0.00,0.00,0.00}{#1}}
\newcommand{\ControlFlowTok}[1]{\textcolor[rgb]{0.13,0.29,0.53}{\textbf{#1}}}
\newcommand{\DataTypeTok}[1]{\textcolor[rgb]{0.13,0.29,0.53}{#1}}
\newcommand{\DecValTok}[1]{\textcolor[rgb]{0.00,0.00,0.81}{#1}}
\newcommand{\DocumentationTok}[1]{\textcolor[rgb]{0.56,0.35,0.01}{\textbf{\textit{#1}}}}
\newcommand{\ErrorTok}[1]{\textcolor[rgb]{0.64,0.00,0.00}{\textbf{#1}}}
\newcommand{\ExtensionTok}[1]{#1}
\newcommand{\FloatTok}[1]{\textcolor[rgb]{0.00,0.00,0.81}{#1}}
\newcommand{\FunctionTok}[1]{\textcolor[rgb]{0.00,0.00,0.00}{#1}}
\newcommand{\ImportTok}[1]{#1}
\newcommand{\InformationTok}[1]{\textcolor[rgb]{0.56,0.35,0.01}{\textbf{\textit{#1}}}}
\newcommand{\KeywordTok}[1]{\textcolor[rgb]{0.13,0.29,0.53}{\textbf{#1}}}
\newcommand{\NormalTok}[1]{#1}
\newcommand{\OperatorTok}[1]{\textcolor[rgb]{0.81,0.36,0.00}{\textbf{#1}}}
\newcommand{\OtherTok}[1]{\textcolor[rgb]{0.56,0.35,0.01}{#1}}
\newcommand{\PreprocessorTok}[1]{\textcolor[rgb]{0.56,0.35,0.01}{\textit{#1}}}
\newcommand{\RegionMarkerTok}[1]{#1}
\newcommand{\SpecialCharTok}[1]{\textcolor[rgb]{0.00,0.00,0.00}{#1}}
\newcommand{\SpecialStringTok}[1]{\textcolor[rgb]{0.31,0.60,0.02}{#1}}
\newcommand{\StringTok}[1]{\textcolor[rgb]{0.31,0.60,0.02}{#1}}
\newcommand{\VariableTok}[1]{\textcolor[rgb]{0.00,0.00,0.00}{#1}}
\newcommand{\VerbatimStringTok}[1]{\textcolor[rgb]{0.31,0.60,0.02}{#1}}
\newcommand{\WarningTok}[1]{\textcolor[rgb]{0.56,0.35,0.01}{\textbf{\textit{#1}}}}
\usepackage{graphicx,grffile}
\makeatletter
\def\maxwidth{\ifdim\Gin@nat@width>\linewidth\linewidth\else\Gin@nat@width\fi}
\def\maxheight{\ifdim\Gin@nat@height>\textheight\textheight\else\Gin@nat@height\fi}
\makeatother
% Scale images if necessary, so that they will not overflow the page
% margins by default, and it is still possible to overwrite the defaults
% using explicit options in \includegraphics[width, height, ...]{}
\setkeys{Gin}{width=\maxwidth,height=\maxheight,keepaspectratio}
\IfFileExists{parskip.sty}{%
\usepackage{parskip}
}{% else
\setlength{\parindent}{0pt}
\setlength{\parskip}{6pt plus 2pt minus 1pt}
}
\setlength{\emergencystretch}{3em}  % prevent overfull lines
\providecommand{\tightlist}{%
  \setlength{\itemsep}{0pt}\setlength{\parskip}{0pt}}
\setcounter{secnumdepth}{0}
% Redefines (sub)paragraphs to behave more like sections
\ifx\paragraph\undefined\else
\let\oldparagraph\paragraph
\renewcommand{\paragraph}[1]{\oldparagraph{#1}\mbox{}}
\fi
\ifx\subparagraph\undefined\else
\let\oldsubparagraph\subparagraph
\renewcommand{\subparagraph}[1]{\oldsubparagraph{#1}\mbox{}}
\fi

%%% Use protect on footnotes to avoid problems with footnotes in titles
\let\rmarkdownfootnote\footnote%
\def\footnote{\protect\rmarkdownfootnote}

%%% Change title format to be more compact
\usepackage{titling}

% Create subtitle command for use in maketitle
\providecommand{\subtitle}[1]{
  \posttitle{
    \begin{center}\large#1\end{center}
    }
}

\setlength{\droptitle}{-2em}

  \title{fav things demo}
    \pretitle{\vspace{\droptitle}\centering\huge}
  \posttitle{\par}
    \author{Jen Richmond}
    \preauthor{\centering\large\emph}
  \postauthor{\par}
      \predate{\centering\large\emph}
  \postdate{\par}
    \date{05/11/2019}


\begin{document}
\maketitle

\hypertarget{my-turn-1}{%
\section{My turn 1}\label{my-turn-1}}

Read and clean fav things data

\hypertarget{load-packages}{%
\subsection{load packages}\label{load-packages}}

\begin{Shaded}
\begin{Highlighting}[]
\KeywordTok{library}\NormalTok{(janitor)}
\KeywordTok{library}\NormalTok{(tidyverse)}
\KeywordTok{library}\NormalTok{(here)}
\KeywordTok{library}\NormalTok{(skimr)}
\end{Highlighting}
\end{Shaded}

\hypertarget{read-data}{%
\subsection{read data}\label{read-data}}

The fav\_things dataset comes from a google form survey I sent around
prior to the workshop. The survey included favourite things ratings and
5 trivia questions about The Sound of Music. You can see a
\href{https://forms.gle/XAhDG1JfAoAnd9G37}{copy of survey here}. For
code to get data from googledrive into R, see the script ``get fav from
googledrive.R''

\begin{Shaded}
\begin{Highlighting}[]
\NormalTok{fav_things <-}\StringTok{ }\KeywordTok{read_csv}\NormalTok{(}\KeywordTok{here}\NormalTok{(}\StringTok{"data_examples"}\NormalTok{, }\StringTok{"favourite_things"}\NormalTok{, }\StringTok{"fav_subset.csv"}\NormalTok{))}
\end{Highlighting}
\end{Shaded}

\begin{verbatim}
## Parsed with column specification:
## cols(
##   Timestamp = col_datetime(format = ""),
##   `Raindrops on roses` = col_double(),
##   `Whiskers on kittens` = col_double(),
##   `Bright copper kettles` = col_double(),
##   `Warm woollen mittens` = col_double(),
##   `Brown paper packages tied up with string` = col_double(),
##   `How old are you?` = col_double(),
##   `How many times have you seen The Sound of Music?` = col_double(),
##   Score = col_double()
## )
\end{verbatim}

\hypertarget{clean-names}{%
\subsection{clean names}\label{clean-names}}

This chunk cleans the variable names, making them all lower case with
underscores in the gaps, creates a new variable called id that creates
ids by row, and drops the timestamp variable using
select(-variablename).

\begin{Shaded}
\begin{Highlighting}[]
\NormalTok{clean <-}\StringTok{ }\NormalTok{fav_things }\OperatorTok
\StringTok{  }\KeywordTok{clean_names}\NormalTok{() }\OperatorTok
\StringTok{  }\KeywordTok{rowid_to_column}\NormalTok{(}\DataTypeTok{var =} \StringTok{"id"}\NormalTok{) }\OperatorTok
\StringTok{  }\KeywordTok{select}\NormalTok{(}\OperatorTok{-}\StringTok{ }\NormalTok{timestamp)}

\CommentTok{#check the names of the variables}
\KeywordTok{names}\NormalTok{(clean)}
\end{Highlighting}
\end{Shaded}

\begin{verbatim}
## [1] "id"                                             
## [2] "raindrops_on_roses"                             
## [3] "whiskers_on_kittens"                            
## [4] "bright_copper_kettles"                          
## [5] "warm_woollen_mittens"                           
## [6] "brown_paper_packages_tied_up_with_string"       
## [7] "how_old_are_you"                                
## [8] "how_many_times_have_you_seen_the_sound_of_music"
## [9] "score"
\end{verbatim}

\hypertarget{rename}{%
\subsection{rename}\label{rename}}

This chunk uses rename() from dplyr to rename each of the variables.

\begin{Shaded}
\begin{Highlighting}[]
\NormalTok{renamed <-}\StringTok{ }\NormalTok{clean }\OperatorTok
\StringTok{  }\KeywordTok{rename}\NormalTok{(}\DataTypeTok{raindrops =}\NormalTok{ raindrops_on_roses, }
         \DataTypeTok{whiskers =}\NormalTok{ whiskers_on_kittens, }
         \DataTypeTok{kettles =}\NormalTok{ bright_copper_kettles, }
         \DataTypeTok{mittens =}\NormalTok{ warm_woollen_mittens, }
         \DataTypeTok{packages =}\NormalTok{ brown_paper_packages_tied_up_with_string, }
         \DataTypeTok{age =}\NormalTok{ how_old_are_you, }
         \DataTypeTok{viewings =}\NormalTok{ how_many_times_have_you_seen_the_sound_of_music)}

\CommentTok{#check names}
\KeywordTok{names}\NormalTok{(renamed)}
\end{Highlighting}
\end{Shaded}

\begin{verbatim}
## [1] "id"        "raindrops" "whiskers"  "kettles"   "mittens"   "packages" 
## [7] "age"       "viewings"  "score"
\end{verbatim}

\hypertarget{my-turn-2-make-it-long}{%
\section{My turn 2 make it long}\label{my-turn-2-make-it-long}}

\hypertarget{make-wide-data-long}{%
\subsection{make wide data long}\label{make-wide-data-long}}

This chunk takes the renamed data and makes the favourite things ratings
from wide to long. The pivot\_longer() function wants to know 3 things.

\begin{enumerate}
\def\labelenumi{\arabic{enumi}.}
\tightlist
\item
  what are the columns that are to be made long
\item
  what is the name of the new column that will contain what are
  currently variable names
\item
  what is the name of the new column that will contain the values.
\end{enumerate}

In this case we want to make the columns raindrops thru packages long,
the names column should be called ``things'', and the values column
should be called ``rating''

\begin{Shaded}
\begin{Highlighting}[]
\NormalTok{long <-}\StringTok{ }\NormalTok{renamed }\OperatorTok
\StringTok{  }\KeywordTok{pivot_longer}\NormalTok{(}\DataTypeTok{cols =}\NormalTok{ raindrops}\OperatorTok{:}\NormalTok{packages, }
               \DataTypeTok{names_to =} \StringTok{"things"}\NormalTok{, }\DataTypeTok{values_to =} \StringTok{"rating"}\NormalTok{) }
\end{Highlighting}
\end{Shaded}

\hypertarget{my-turn-3--summarise-fav-things}{%
\section{My turn 3- summarise fav
things}\label{my-turn-3--summarise-fav-things}}

\hypertarget{look-at-the-structure-of-the-data}{%
\subsection{look at the structure of the
data}\label{look-at-the-structure-of-the-data}}

The head() function shows you just the first 6 rows of the dataframe. It
is a useful sanity check when you are checking that the data you have
read into R looks as you expect.

\begin{Shaded}
\begin{Highlighting}[]
\KeywordTok{head}\NormalTok{(long)}
\end{Highlighting}
\end{Shaded}

\begin{verbatim}
## # A tibble: 6 x 6
##      id   age viewings score things    rating
##   <int> <dbl>    <dbl> <dbl> <chr>      <dbl>
## 1     1    42       17     5 raindrops      5
## 2     1    42       17     5 whiskers       3
## 3     1    42       17     5 kettles        4
## 4     1    42       17     5 mittens        2
## 5     1    42       17     5 packages       4
## 6     2    32        2     4 raindrops      6
\end{verbatim}

The str() function gives you an idea of the ``structure'' of the
dataframe. It prints how many observations and variables you have and
what kind of data (int, num, chr) is in each variable.

\begin{Shaded}
\begin{Highlighting}[]
\KeywordTok{str}\NormalTok{(long)}
\end{Highlighting}
\end{Shaded}

\begin{verbatim}
## Classes 'tbl_df', 'tbl' and 'data.frame':    495 obs. of  6 variables:
##  $ id      : int  1 1 1 1 1 2 2 2 2 2 ...
##  $ age     : num  42 42 42 42 42 32 32 32 32 32 ...
##  $ viewings: num  17 17 17 17 17 2 2 2 2 2 ...
##  $ score   : num  5 5 5 5 5 4 4 4 4 4 ...
##  $ things  : chr  "raindrops" "whiskers" "kettles" "mittens" ...
##  $ rating  : num  5 3 4 2 4 6 6 4 4 4 ...
\end{verbatim}

The glimpse() function gives the same info as str() and the formatting
is a bit nicer. So it is my go-to.

\begin{Shaded}
\begin{Highlighting}[]
\KeywordTok{glimpse}\NormalTok{(long)}
\end{Highlighting}
\end{Shaded}

\begin{verbatim}
## Observations: 495
## Variables: 6
## $ id       <int> 1, 1, 1, 1, 1, 2, 2, 2, 2, 2, 3, 3, 3, 3, 3, 4, 4, 4,...
## $ age      <dbl> 42, 42, 42, 42, 42, 32, 32, 32, 32, 32, 31, 31, 31, 3...
## $ viewings <dbl> 17, 17, 17, 17, 17, 2, 2, 2, 2, 2, 3, 3, 3, 3, 3, 1, ...
## $ score    <dbl> 5, 5, 5, 5, 5, 4, 4, 4, 4, 4, 2, 2, 2, 2, 2, 3, 3, 3,...
## $ things   <chr> "raindrops", "whiskers", "kettles", "mittens", "packa...
## $ rating   <dbl> 5, 3, 4, 2, 4, 6, 6, 4, 4, 4, 4, 7, 6, 3, 1, 5, 7, 3,...
\end{verbatim}

\hypertarget{get-summary-stats}{%
\subsection{get summary stats}\label{get-summary-stats}}

There a few functions in R that are useful to get quick and dirty
summary data.

The summary() function gives you means for numeric variables, but isn't
terribly useful for other kinds of data.

\begin{Shaded}
\begin{Highlighting}[]
\KeywordTok{summary}\NormalTok{(long)}
\end{Highlighting}
\end{Shaded}

\begin{verbatim}
##        id          age           viewings           score     
##  Min.   : 1   Min.   : 5.00   Min.   :  0.000   Min.   :0.00  
##  1st Qu.:25   1st Qu.:34.00   1st Qu.:  2.000   1st Qu.:2.00  
##  Median :50   Median :40.00   Median :  4.000   Median :3.00  
##  Mean   :50   Mean   :37.98   Mean   :  9.576   Mean   :2.99  
##  3rd Qu.:75   3rd Qu.:44.00   3rd Qu.: 10.000   3rd Qu.:4.00  
##  Max.   :99   Max.   :67.00   Max.   :150.000   Max.   :5.00  
##               NA's   :5                                       
##     things              rating     
##  Length:495         Min.   :1.000  
##  Class :character   1st Qu.:4.000  
##  Mode  :character   Median :5.000  
##                     Mean   :4.897  
##                     3rd Qu.:7.000  
##                     Max.   :7.000  
##                     NA's   :2
\end{verbatim}

The skim() function from the skimr package groups variables by type and
gives useful summary stats for each kind of variable. It also includes a
mini histogram for numeric variables.

\begin{Shaded}
\begin{Highlighting}[]
\KeywordTok{skim}\NormalTok{(long) }
\end{Highlighting}
\end{Shaded}

\begin{verbatim}
## Skim summary statistics
##  n obs: 495 
##  n variables: 6 
## 
## -- Variable type:character ---------------------------------------
##  variable missing complete   n min max empty n_unique
##    things       0      495 495   7   9     0        5
## 
## -- Variable type:integer -----------------------------------------
##  variable missing complete   n mean    sd p0 p25 p50 p75 p100     hist
##        id       0      495 495   50 28.61  1  25  50  75   99 ▇▇▇▇▇▇▇▇
## 
## -- Variable type:numeric -----------------------------------------
##  variable missing complete   n  mean    sd p0 p25 p50 p75 p100     hist
##       age       5      490 495 37.98 12.64  5  34  40  44   67 ▁▁▂▂▇▃▁▁
##    rating       2      493 495  4.9   1.83  1   4   5   7    7 ▂▂▃▃▁▇▆▇
##     score       0      495 495  2.99  1.29  0   2   3   4    5 ▁▃▁▅▇▁▃▃
##  viewings       0      495 495  9.58 18.06  0   2   4  10  150 ▇▁▁▁▁▁▁▁
\end{verbatim}

If you want summary stats that you can work with (i.e.~plot), you need
to create a dataframe. This chuck takes the long dataframe, pipes it
into a group\_by() function, which allows you to group the subsqeunt
stats by a variable of interest (in this case ``things''), then pipes
that into a summarise function to calculate the mean, standard
deviation, n, and standard error. The output of that is saved to a
dataframe called ratings\_summary. This summary dataframe can then be
used as a input to a ggplot.

\begin{Shaded}
\begin{Highlighting}[]
\NormalTok{ratings_summary <-}\StringTok{ }\NormalTok{long }\OperatorTok
\StringTok{  }\KeywordTok{group_by}\NormalTok{(things) }\OperatorTok
\StringTok{  }\KeywordTok{summarise}\NormalTok{(}\DataTypeTok{mean =} \KeywordTok{mean}\NormalTok{(rating, }\DataTypeTok{na.rm =} \OtherTok{TRUE}\NormalTok{), }
            \DataTypeTok{sd =} \KeywordTok{sd}\NormalTok{(rating, }\DataTypeTok{na.rm =} \OtherTok{TRUE}\NormalTok{), }
            \DataTypeTok{n =} \KeywordTok{n}\NormalTok{(), }
            \DataTypeTok{se =}\NormalTok{ sd}\OperatorTok{/}\KeywordTok{sqrt}\NormalTok{(n)) }
\end{Highlighting}
\end{Shaded}

\hypertarget{my-turn-4-plot-fav-things}{%
\section{My turn 4: plot fav things}\label{my-turn-4-plot-fav-things}}

\hypertarget{plot-raw-points}{%
\subsection{plot raw points}\label{plot-raw-points}}

One of my favourite things abotu ggplot is being able to plot scores
from individual participants (or even individual trials) really easily.
It is a great way to get a feel for the variability in your data and to
detect mistakes.

Here we take the long dataframe and pipe it into ggplot, setting age to
appear on the x axis and viewings to appear on the y axis. Then we add a
geom layer with points.

\begin{Shaded}
\begin{Highlighting}[]
\NormalTok{long }\OperatorTok
\StringTok{  }\KeywordTok{ggplot}\NormalTok{(}\KeywordTok{aes}\NormalTok{(}\DataTypeTok{x =}\NormalTok{ age, }\DataTypeTok{y =}\NormalTok{ viewings)) }\OperatorTok{+}
\StringTok{  }\KeywordTok{geom_point}\NormalTok{() }
\end{Highlighting}
\end{Shaded}

\begin{verbatim}
## Warning: Removed 5 rows containing missing values (geom_point).
\end{verbatim}

\includegraphics{fav_demo_files/figure-latex/unnamed-chunk-12-1.pdf}

\hypertarget{plot-summary---simple-plot}{%
\subsection{plot summary - simple
plot}\label{plot-summary---simple-plot}}

Here we take the summary dataframe we created above, pipe it into
ggplot, put things on the x axis and mean on the y axis and add a geom
layer with columns.

\begin{Shaded}
\begin{Highlighting}[]
\NormalTok{ratings_summary }\OperatorTok
\StringTok{  }\KeywordTok{ggplot}\NormalTok{(}\KeywordTok{aes}\NormalTok{(}\DataTypeTok{x =}\NormalTok{ things, }\DataTypeTok{y =}\NormalTok{ mean)) }\OperatorTok{+}
\StringTok{  }\KeywordTok{geom_col}\NormalTok{()}
\end{Highlighting}
\end{Shaded}

\includegraphics{fav_demo_files/figure-latex/unnamed-chunk-13-1.pdf}

\hypertarget{plot-summary---bells-and-whistles-plot}{%
\subsection{plot summary - bells and whistles
plot}\label{plot-summary---bells-and-whistles-plot}}

This chunk takes code from above and fancies it up a bit! Adding error
bars, colour, fixing the axis labels etc.

\begin{Shaded}
\begin{Highlighting}[]
\NormalTok{long}\OperatorTok{$}\NormalTok{score <-}\StringTok{ }\KeywordTok{as.factor}\NormalTok{(long}\OperatorTok{$}\NormalTok{score) }\CommentTok{# make score a factor so the colours work}

\NormalTok{ratings_summary }\OperatorTok
\StringTok{  }\KeywordTok{ggplot}\NormalTok{(}\KeywordTok{aes}\NormalTok{(}\DataTypeTok{x =}\NormalTok{ things, }\DataTypeTok{y =}\NormalTok{ mean, }\DataTypeTok{fill =}\NormalTok{ things)) }\OperatorTok{+}
\StringTok{  }\KeywordTok{geom_col}\NormalTok{() }\OperatorTok{+}
\StringTok{  }\KeywordTok{geom_errorbar}\NormalTok{(}\KeywordTok{aes}\NormalTok{(}\DataTypeTok{ymin=}\NormalTok{ mean}\OperatorTok{-}\NormalTok{se, }\DataTypeTok{ymax=}\NormalTok{mean}\OperatorTok{+}\NormalTok{se),}
                \DataTypeTok{size=} \FloatTok{0.3}\NormalTok{,    }\CommentTok{# thinner lines}
                \DataTypeTok{width=} \FloatTok{0.2}\NormalTok{)  }\OperatorTok{+}\StringTok{  }\CommentTok{# skinnier bars }
\StringTok{  }\KeywordTok{theme_classic}\NormalTok{() }\OperatorTok{+}\StringTok{ }\CommentTok{# white background APA style}
\StringTok{  }\KeywordTok{theme}\NormalTok{(}\DataTypeTok{legend.position =} \StringTok{"none"}\NormalTok{) }\OperatorTok{+}\StringTok{ }\CommentTok{#remove legend}
\StringTok{  }\KeywordTok{scale_y_continuous}\NormalTok{(}\DataTypeTok{name =} \StringTok{"mean rating"}\NormalTok{, }
                     \DataTypeTok{expand =} \KeywordTok{c}\NormalTok{(}\DecValTok{0}\NormalTok{, }\DecValTok{0}\NormalTok{), }\CommentTok{# make the bars sit on x axis}
                     \DataTypeTok{limits =} \KeywordTok{c}\NormalTok{(}\DecValTok{0}\NormalTok{, }\DecValTok{7}\NormalTok{), }\CommentTok{# extend y axis}
                     \DataTypeTok{breaks=}\KeywordTok{c}\NormalTok{(}\DecValTok{0}\NormalTok{, }\DecValTok{1}\NormalTok{, }\DecValTok{2}\NormalTok{, }\DecValTok{3}\NormalTok{, }\DecValTok{4}\NormalTok{, }\DecValTok{5}\NormalTok{, }\DecValTok{6}\NormalTok{, }\DecValTok{7}\NormalTok{))  }\CommentTok{# make the ticks 1}
\end{Highlighting}
\end{Shaded}

\includegraphics{fav_demo_files/figure-latex/unnamed-chunk-14-1.pdf}


\end{document}
